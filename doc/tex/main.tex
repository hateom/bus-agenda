\section{Sformułowanie zadania projektowego}
Podanie przedmiotu projektowania, jego celów,
przeglądu zadań, specyfiki i uwarunkowań.
\section{Analiza stanu wyjściowego}
Analiza stanu zastanego, uwarunkowań prawnych, przyjętego
obiegu istniejącej dokumentacji, analiza istniejącego systemu elektronicznego
przetwarzania danych (aktualnej bazy), analiza występujących problemów, etc. pomocne
mogą być scenariusze postępowania i ich analiza (elementy, obiekty, charakterystyki,
atrybuty, struktura, przepływ danych, powiązania, relacje, ograniczenia funkcjonalności).
\section{Analiza wymagań użytkownika}
Na tym etapie należy określić podstawowe
cele, zadania i funkcjonalność jakie mają być realizowane przez projektowaną bazę danych
oraz ew. wymagania dotyczące projektu i dokumentacji. Dobrze byłoby, aby użytkownik
na bieżąco współuczestniczył w projektowaniu i implementacji oraz wnosił swoje uwagi.
Należy zidentyfikować wymagania jawne i niejawne.
\section{Określenie scenariuszy użycia}
Scenariusze użycia pozwolą na konstrukcję diagramów
DFD i STD oraz hierarchii funkcji.
\section{Identyfikacja funkcji}
Określenie podstawowych funkcji realizowanych w bazie danych.
\section{Analiza hierarchii funkcji projektowanej aplikacji (FHD - Functional Hierarchy
Diagram)}
Określenie struktury zależności hierarchicznych pomiędzy jednostkami
analizowanego systemu, zwłaszcza w zakresie specyfikacji wymagań funkcjonalnych.
Specyfikacja funkcji (funkcjonalności) projektowanego systemu.
\section{Budowa i analiza diagramu przepływu danych - Data Flow Diagram }
Ma na celu określenie przepływu danych (wejścia, wyjścia, operacje, przechowywanie) oraz
elementów sterowania tym przepływem, co może być pomocne dla tworzenia aplikacji.
Specyfikacja danych wejściowych i wyjściowych.
\section{Wybór encji (obiektów) i ich atrybutów}
\section{Projektowanie powiązań (relacji) pomiędzy encjami}
Konstrukcja diagramu ERD
(Entity-Relationship Diagram); jest to zasadniczy etap procesu projektowania struktury
bazy danych. Identyfikacja klas encji, ich atrybutów, zdefiniowanie (określenie) kluczy.
Tablica krzyżowa powiązań, eliminacja powiązań wiele-do-wielu. Konstrukcja diagramu
ERD.
\section{Projekt diagramów STD (State Transition Diagram - diagramy przejść pomiędzy
stanami)}
Wykonanie w oparciu o scenariusze użycia i strukturę bazy danych. Pomocny do
budowy interfejsu aplikacji.
