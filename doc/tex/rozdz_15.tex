\section{Projektowanie operacji na danych: zdefiniowanie kwerend dla realizacji funkcji wyspecyfikowanych w projekcie}

Przykładowe zapytania SQL:
Wyświetlenie wszystkich odjazdów ze wszystkich przystanków dla trasy o ID=1 o wszystkich porach dnia (oraz pokazuje te pory). Docelowo będzie konieczne sprarsowanie takiego wyniku, aby wyświetlić go w bardziej przystępnej formie.
\begin{verbatim}
SELECT (SELECT pory.opis FROM pory WHERE pory.id = przesuniecia.pory_id) AS "Pora", przystanki.nazwa AS Przystanek, przesuniecia.offset + odjazdy.godzina AS Odj FROM przystanki INNER JOIN przesuniecia ON (linie_id=1 AND przystanki_id=przystanki.id) INNER JOIN odjazdy ON odjazdy.linie_id=1 ORDER BY przesuniecia.numer_kolejny ASC, przesuniecia.pory_id ASC;
\end{verbatim}

Wyświetlenie wszystkich dostępnych linii autobusowych:
\begin{verbatim}
SELECT linie.numer AS "Numer", linie.opis AS "Opis" FROM linie;
\end{verbatim}


Wyświetlenie dostępnych przystanków wraz z nazwą ulic przy której się znajdują:
\begin{verbatim}
SELECT przystanki.nazwa AS "Nazwa przystanku" , (SELECT ulice_d.nazwa FROM ulice_d WHERE ulice_d.id=przystanki.ulice_d_id) AS "Ulica" FROM przystanki;
\end{verbatim}

Wyświetlenie wszystkich pór dnia wraz z ich opisem
\begin{verbatim}
SELECT pory.opis AS "Opis", pory.rozp || ' - ' || pory.zakoncz AS "Odjazdy" FROM pory;
\end{verbatim}


Przykładowe dodanie nowej linii:
\begin{verbatim}
INSERT INTO linie VALUES (0, 194, 1, 'Stacja A- Stacja B');
\end{verbatim}

Są to oczywiście jedynie przykładowe zapytanie, pokazujące, że rozkład jazdy działa i możliwa jest jego generacja. Dodatkowo widać, że powiązania pomiędzy tabelami sa poprawne (dla przykładowych danych otrzymane wyniki są poprawne). W ostatecznej wersji zapytania te mogą wyglądać nieco inaczej choćby ze względu na konieczność ich dopasowania do wymagań GUI.

Dodatkowo utworzone zostaną widoki z wygenerowanymi już rozkładami dla każdego przystanku, zależne od pory dnia. Jest to konieczne ze względu na wydajność (chcemy uniknąć przeliczania rozkładu przy każdym żądaniu).