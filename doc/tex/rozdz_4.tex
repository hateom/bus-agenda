\section{Określenie scenariuszy użycia}
% \textit{Scenariusze użycia pozwolą na konstrukcję diagramów
% DFD i STD oraz hierarchii funkcji.}
W projekcie przewidujemy dwa główne scenariusze użytkowania : użytkownik i administrator.\\
Użytkownik będzie mógł:
\begin{enumerate}
\item poszukiwać połączeń, w naszym wypadku dokonamy uproszczenia (poszukiwać będziemy tylko pośród linii wyjeżdżających z przystanku źródłowego i dojeżdżających do punktu docelowego), ponieważ pełne wyszukiwanie połączeń (takie jak np. na stronie MPK) 
jest zbyt skomplikowane i nie jest możliwe za pomocą podstawowych operacji na bazie danych
\item poszukiwać przystanków, według nazw
\item wyświetlić dowolną linię wraz z pokazaniem listy przystanków i wyborem kierunku jazdy
\item poszukiwać przystanków w pobliży zadanej ulicy
\item wyświetlać wybrany rozkład jazdy (dla przystanku)
\end{enumerate}

Administrator będzie odpowiedzialny za dodawanie, edycję oraz usuwanie danych z rozkładu tzn:
\begin{enumerate}
\item dodawać nowe przystanki, edytować i usuwać już istniejące. Dodawanie polegać będzie na określeniu nazwy przystanku oraz ulic będących w jego pobliżu.
Usuwanie przystanku będzie możliwe jedynie pod warunkiem wprowadzenia korekty w rozkładach jazdy (tj upewnienie się, że żadna z linii nie
jeździ przez usuwany przystanek)
\item dodawać linie  (przystanek po przystanku) i określać czas podróży pomiędzy przystankami,
\item usuwać linie,
\item modyfikować istniejące trasy oraz generować i modyfikować rozkłady jazdy
\end{enumerate}


