\section{Analiza stanu wyjściowego}


%\textit{Analiza stanu zastanego, uwarunkowań prawnych, przyjętego
%obiegu istniejącej dokumentacji, analiza istniejącego systemu elektronicznego
%przetwarzania danych (aktualnej bazy), analiza występujących problemów, etc. pomocne
%mogą być scenariusze postępowania i ich analiza (elementy, obiekty, charakterystyki,
%atrybuty, struktura, przepływ danych, powiązania, relacje, ograniczenia funkcjonalności).} \\
Nasz system budujemy od podstaw, więc nie mamy żadnego istniejącego systemu bądź dokumentacji. Oczywiście istnieją podobne rozwiązania,
jak np. strona krakowskiego MPK. Rozkład jazdy udostępniony publicznie przez MPK umożliwia wyszukiwanie przystanków według nazw ulic, nazw przystanków, oraz pokazywanie tras z możliwością wyboru dowolnego z przystanków i pokazania jego rozkładu jazdy. Strona wykonana jest w technologii HTML (strony są statyczne, najprawdopodobniej wygenerowane programem). Możliwe jest też zaawansowane wyszukiwanie połączeń (jest to z kolei skrypt CGI). Naszym celem jest osiągnięcie podobnej funkcjonalności. Niestety, z powodu pewnych ograniczeń naszej bazy nie możemy na tym etapie wykonać pełnej implementacji wyszukiwarki połączeń.